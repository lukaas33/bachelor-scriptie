Based on the definitions of the relations, some relations are redundant. 
These relations can be pruned from the graph without loss of information since they are implicit from other relations. 
This pruning process results in a more sparse graph which is more easily interpreted by humans.
The instances in which relations can be pruned are explained in Figure~\ref{fig:Pruning}.
%
% TODO generate nicer images
%
\begin{figure}
    % TODO add explanation
    \newcommand{\subfigurelength}{0.35\textwidth} % TODO move this command
    \centering
    \begin{subfigure}{\subfigurelength}
        \centering
        \includesvg[width=\textwidth]{relations/pruning/reflexive.svg}
        \caption{Reflexivity. The equivalence relation is reflexive. This means that every variant is equivalent to itself. This is understood implicitly and thus doesn't have to be represented in the graph.}
        \label{sfig:Reflexivity}
    \end{subfigure}
    \begin{subfigure}{\subfigurelength}
        \centering
        \includesvg[width=\textwidth]{relations/pruning/symmetric.svg}
        \caption{Symmetry. The equi}
        \label{sfig:Symmetry}
    \end{subfigure}
    \begin{subfigure}{\subfigurelength}
        \centering
        \includesvg[width=\textwidth]{relations/pruning/transitive.svg}
        \caption{Transitivity}
        \label{sfig:Transitivity}
    \end{subfigure}
    \begin{subfigure}{\subfigurelength}
        \centering
        \includesvg[width=\textwidth]{relations/pruning/common-ancestor.svg}
        \caption{Common Ancestor}
        \label{sfig:Common-Ancestor}
    \end{subfigure}
    \begin{subfigure}{\subfigurelength}
        \centering
        \includesvg[width=\textwidth]{relations/pruning/most-specific.svg}
        \caption{Most Specific}
        \label{sfig:Most-Specific}
    \end{subfigure}
    % TODO reduce equivalence relations
    \caption{Different instances in which relations can be pruned.}
    \label{fig:Pruning}
\end{figure}
