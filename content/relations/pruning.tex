Based on the definitions of the relations, some relations are redundant. 
These relations can be pruned from the graph without loss of information since they are implicit from other relations. 
This pruning process results in a more sparse graph which is more easily interpreted by humans.
The instances in which relations can be pruned are explained in Figure~\ref{fig:Pruning}.
%
% TODO generate nicer images
%
\begin{figure}
    % TODO add explanation
    \newcommand{\subfigurelength}{0.35\textwidth} % TODO move command
    \centering
    \begin{subfigure}{\subfigurelength}
        \centering
        \includesvg[width=\textwidth]{relations/pruning/reflexive.svg}
        \caption{Reflexivity}
        \label{sfig:Reflexivity}
    \end{subfigure}
    \begin{subfigure}{\subfigurelength}
        \centering
        \includesvg[width=\textwidth]{relations/pruning/symmetric.svg}
        \caption{Symmetric}
        \label{sfig:Symmetry}
    \end{subfigure}
    \begin{subfigure}{\subfigurelength}
        \centering
        \includesvg[width=\textwidth]{relations/pruning/transitive.svg}
        \caption{Transitivity}
        \label{sfig:Transitivity}
    \end{subfigure}
    \begin{subfigure}{\subfigurelength}
        \centering
        \includesvg[width=\textwidth]{relations/pruning/common-ancestor.svg}
        \caption{Common Ancestor}
        \label{sfig:Common-Ancestor}
    \end{subfigure}
    \begin{subfigure}{\subfigurelength}
        \centering
        \includesvg[width=\textwidth]{relations/pruning/most-specific.svg}
        \caption{Most Specific}
        \label{sfig:Most-Specific}
    \end{subfigure}
    \caption{Different instances in which relations can be pruned.}
    \label{fig:Pruning}
\end{figure}
